
% Default to the notebook output style

    


% Inherit from the specified cell style.




    
\documentclass[11pt]{article}

    
    
    \usepackage[T1]{fontenc}
    % Nicer default font (+ math font) than Computer Modern for most use cases
    \usepackage{mathpazo}

    % Basic figure setup, for now with no caption control since it's done
    % automatically by Pandoc (which extracts ![](path) syntax from Markdown).
    \usepackage{graphicx}
    % We will generate all images so they have a width \maxwidth. This means
    % that they will get their normal width if they fit onto the page, but
    % are scaled down if they would overflow the margins.
    \makeatletter
    \def\maxwidth{\ifdim\Gin@nat@width>\linewidth\linewidth
    \else\Gin@nat@width\fi}
    \makeatother
    \let\Oldincludegraphics\includegraphics
    % Set max figure width to be 80% of text width, for now hardcoded.
    \renewcommand{\includegraphics}[1]{\Oldincludegraphics[width=.8\maxwidth]{#1}}
    % Ensure that by default, figures have no caption (until we provide a
    % proper Figure object with a Caption API and a way to capture that
    % in the conversion process - todo).
    \usepackage{caption}
    \DeclareCaptionLabelFormat{nolabel}{}
    \captionsetup{labelformat=nolabel}

    \usepackage{adjustbox} % Used to constrain images to a maximum size 
    \usepackage{xcolor} % Allow colors to be defined
    \usepackage{enumerate} % Needed for markdown enumerations to work
    \usepackage{geometry} % Used to adjust the document margins
    \usepackage{amsmath} % Equations
    \usepackage{amssymb} % Equations
    \usepackage{textcomp} % defines textquotesingle
    % Hack from http://tex.stackexchange.com/a/47451/13684:
    \AtBeginDocument{%
        \def\PYZsq{\textquotesingle}% Upright quotes in Pygmentized code
    }
    \usepackage{upquote} % Upright quotes for verbatim code
    \usepackage{eurosym} % defines \euro
    \usepackage[mathletters]{ucs} % Extended unicode (utf-8) support
    \usepackage[utf8x]{inputenc} % Allow utf-8 characters in the tex document
    \usepackage{fancyvrb} % verbatim replacement that allows latex
    \usepackage{grffile} % extends the file name processing of package graphics 
                         % to support a larger range 
    % The hyperref package gives us a pdf with properly built
    % internal navigation ('pdf bookmarks' for the table of contents,
    % internal cross-reference links, web links for URLs, etc.)
    \usepackage{hyperref}
    \usepackage{longtable} % longtable support required by pandoc >1.10
    \usepackage{booktabs}  % table support for pandoc > 1.12.2
    \usepackage[inline]{enumitem} % IRkernel/repr support (it uses the enumerate* environment)
    \usepackage[normalem]{ulem} % ulem is needed to support strikethroughs (\sout)
                                % normalem makes italics be italics, not underlines
    

    
    
    % Colors for the hyperref package
    \definecolor{urlcolor}{rgb}{0,.145,.698}
    \definecolor{linkcolor}{rgb}{.71,0.21,0.01}
    \definecolor{citecolor}{rgb}{.12,.54,.11}

    % ANSI colors
    \definecolor{ansi-black}{HTML}{3E424D}
    \definecolor{ansi-black-intense}{HTML}{282C36}
    \definecolor{ansi-red}{HTML}{E75C58}
    \definecolor{ansi-red-intense}{HTML}{B22B31}
    \definecolor{ansi-green}{HTML}{00A250}
    \definecolor{ansi-green-intense}{HTML}{007427}
    \definecolor{ansi-yellow}{HTML}{DDB62B}
    \definecolor{ansi-yellow-intense}{HTML}{B27D12}
    \definecolor{ansi-blue}{HTML}{208FFB}
    \definecolor{ansi-blue-intense}{HTML}{0065CA}
    \definecolor{ansi-magenta}{HTML}{D160C4}
    \definecolor{ansi-magenta-intense}{HTML}{A03196}
    \definecolor{ansi-cyan}{HTML}{60C6C8}
    \definecolor{ansi-cyan-intense}{HTML}{258F8F}
    \definecolor{ansi-white}{HTML}{C5C1B4}
    \definecolor{ansi-white-intense}{HTML}{A1A6B2}

    % commands and environments needed by pandoc snippets
    % extracted from the output of `pandoc -s`
    \providecommand{\tightlist}{%
      \setlength{\itemsep}{0pt}\setlength{\parskip}{0pt}}
    \DefineVerbatimEnvironment{Highlighting}{Verbatim}{commandchars=\\\{\}}
    % Add ',fontsize=\small' for more characters per line
    \newenvironment{Shaded}{}{}
    \newcommand{\KeywordTok}[1]{\textcolor[rgb]{0.00,0.44,0.13}{\textbf{{#1}}}}
    \newcommand{\DataTypeTok}[1]{\textcolor[rgb]{0.56,0.13,0.00}{{#1}}}
    \newcommand{\DecValTok}[1]{\textcolor[rgb]{0.25,0.63,0.44}{{#1}}}
    \newcommand{\BaseNTok}[1]{\textcolor[rgb]{0.25,0.63,0.44}{{#1}}}
    \newcommand{\FloatTok}[1]{\textcolor[rgb]{0.25,0.63,0.44}{{#1}}}
    \newcommand{\CharTok}[1]{\textcolor[rgb]{0.25,0.44,0.63}{{#1}}}
    \newcommand{\StringTok}[1]{\textcolor[rgb]{0.25,0.44,0.63}{{#1}}}
    \newcommand{\CommentTok}[1]{\textcolor[rgb]{0.38,0.63,0.69}{\textit{{#1}}}}
    \newcommand{\OtherTok}[1]{\textcolor[rgb]{0.00,0.44,0.13}{{#1}}}
    \newcommand{\AlertTok}[1]{\textcolor[rgb]{1.00,0.00,0.00}{\textbf{{#1}}}}
    \newcommand{\FunctionTok}[1]{\textcolor[rgb]{0.02,0.16,0.49}{{#1}}}
    \newcommand{\RegionMarkerTok}[1]{{#1}}
    \newcommand{\ErrorTok}[1]{\textcolor[rgb]{1.00,0.00,0.00}{\textbf{{#1}}}}
    \newcommand{\NormalTok}[1]{{#1}}
    
    % Additional commands for more recent versions of Pandoc
    \newcommand{\ConstantTok}[1]{\textcolor[rgb]{0.53,0.00,0.00}{{#1}}}
    \newcommand{\SpecialCharTok}[1]{\textcolor[rgb]{0.25,0.44,0.63}{{#1}}}
    \newcommand{\VerbatimStringTok}[1]{\textcolor[rgb]{0.25,0.44,0.63}{{#1}}}
    \newcommand{\SpecialStringTok}[1]{\textcolor[rgb]{0.73,0.40,0.53}{{#1}}}
    \newcommand{\ImportTok}[1]{{#1}}
    \newcommand{\DocumentationTok}[1]{\textcolor[rgb]{0.73,0.13,0.13}{\textit{{#1}}}}
    \newcommand{\AnnotationTok}[1]{\textcolor[rgb]{0.38,0.63,0.69}{\textbf{\textit{{#1}}}}}
    \newcommand{\CommentVarTok}[1]{\textcolor[rgb]{0.38,0.63,0.69}{\textbf{\textit{{#1}}}}}
    \newcommand{\VariableTok}[1]{\textcolor[rgb]{0.10,0.09,0.49}{{#1}}}
    \newcommand{\ControlFlowTok}[1]{\textcolor[rgb]{0.00,0.44,0.13}{\textbf{{#1}}}}
    \newcommand{\OperatorTok}[1]{\textcolor[rgb]{0.40,0.40,0.40}{{#1}}}
    \newcommand{\BuiltInTok}[1]{{#1}}
    \newcommand{\ExtensionTok}[1]{{#1}}
    \newcommand{\PreprocessorTok}[1]{\textcolor[rgb]{0.74,0.48,0.00}{{#1}}}
    \newcommand{\AttributeTok}[1]{\textcolor[rgb]{0.49,0.56,0.16}{{#1}}}
    \newcommand{\InformationTok}[1]{\textcolor[rgb]{0.38,0.63,0.69}{\textbf{\textit{{#1}}}}}
    \newcommand{\WarningTok}[1]{\textcolor[rgb]{0.38,0.63,0.69}{\textbf{\textit{{#1}}}}}
    
    
    % Define a nice break command that doesn't care if a line doesn't already
    % exist.
    \def\br{\hspace*{\fill} \\* }
    % Math Jax compatability definitions
    \def\gt{>}
    \def\lt{<}
    % Document parameters
    \title{TMDb Movie Data}
    
    
    

    % Pygments definitions
    
\makeatletter
\def\PY@reset{\let\PY@it=\relax \let\PY@bf=\relax%
    \let\PY@ul=\relax \let\PY@tc=\relax%
    \let\PY@bc=\relax \let\PY@ff=\relax}
\def\PY@tok#1{\csname PY@tok@#1\endcsname}
\def\PY@toks#1+{\ifx\relax#1\empty\else%
    \PY@tok{#1}\expandafter\PY@toks\fi}
\def\PY@do#1{\PY@bc{\PY@tc{\PY@ul{%
    \PY@it{\PY@bf{\PY@ff{#1}}}}}}}
\def\PY#1#2{\PY@reset\PY@toks#1+\relax+\PY@do{#2}}

\expandafter\def\csname PY@tok@gd\endcsname{\def\PY@tc##1{\textcolor[rgb]{0.63,0.00,0.00}{##1}}}
\expandafter\def\csname PY@tok@gu\endcsname{\let\PY@bf=\textbf\def\PY@tc##1{\textcolor[rgb]{0.50,0.00,0.50}{##1}}}
\expandafter\def\csname PY@tok@gt\endcsname{\def\PY@tc##1{\textcolor[rgb]{0.00,0.27,0.87}{##1}}}
\expandafter\def\csname PY@tok@gs\endcsname{\let\PY@bf=\textbf}
\expandafter\def\csname PY@tok@gr\endcsname{\def\PY@tc##1{\textcolor[rgb]{1.00,0.00,0.00}{##1}}}
\expandafter\def\csname PY@tok@cm\endcsname{\let\PY@it=\textit\def\PY@tc##1{\textcolor[rgb]{0.25,0.50,0.50}{##1}}}
\expandafter\def\csname PY@tok@vg\endcsname{\def\PY@tc##1{\textcolor[rgb]{0.10,0.09,0.49}{##1}}}
\expandafter\def\csname PY@tok@vi\endcsname{\def\PY@tc##1{\textcolor[rgb]{0.10,0.09,0.49}{##1}}}
\expandafter\def\csname PY@tok@vm\endcsname{\def\PY@tc##1{\textcolor[rgb]{0.10,0.09,0.49}{##1}}}
\expandafter\def\csname PY@tok@mh\endcsname{\def\PY@tc##1{\textcolor[rgb]{0.40,0.40,0.40}{##1}}}
\expandafter\def\csname PY@tok@cs\endcsname{\let\PY@it=\textit\def\PY@tc##1{\textcolor[rgb]{0.25,0.50,0.50}{##1}}}
\expandafter\def\csname PY@tok@ge\endcsname{\let\PY@it=\textit}
\expandafter\def\csname PY@tok@vc\endcsname{\def\PY@tc##1{\textcolor[rgb]{0.10,0.09,0.49}{##1}}}
\expandafter\def\csname PY@tok@il\endcsname{\def\PY@tc##1{\textcolor[rgb]{0.40,0.40,0.40}{##1}}}
\expandafter\def\csname PY@tok@go\endcsname{\def\PY@tc##1{\textcolor[rgb]{0.53,0.53,0.53}{##1}}}
\expandafter\def\csname PY@tok@cp\endcsname{\def\PY@tc##1{\textcolor[rgb]{0.74,0.48,0.00}{##1}}}
\expandafter\def\csname PY@tok@gi\endcsname{\def\PY@tc##1{\textcolor[rgb]{0.00,0.63,0.00}{##1}}}
\expandafter\def\csname PY@tok@gh\endcsname{\let\PY@bf=\textbf\def\PY@tc##1{\textcolor[rgb]{0.00,0.00,0.50}{##1}}}
\expandafter\def\csname PY@tok@ni\endcsname{\let\PY@bf=\textbf\def\PY@tc##1{\textcolor[rgb]{0.60,0.60,0.60}{##1}}}
\expandafter\def\csname PY@tok@nl\endcsname{\def\PY@tc##1{\textcolor[rgb]{0.63,0.63,0.00}{##1}}}
\expandafter\def\csname PY@tok@nn\endcsname{\let\PY@bf=\textbf\def\PY@tc##1{\textcolor[rgb]{0.00,0.00,1.00}{##1}}}
\expandafter\def\csname PY@tok@no\endcsname{\def\PY@tc##1{\textcolor[rgb]{0.53,0.00,0.00}{##1}}}
\expandafter\def\csname PY@tok@na\endcsname{\def\PY@tc##1{\textcolor[rgb]{0.49,0.56,0.16}{##1}}}
\expandafter\def\csname PY@tok@nb\endcsname{\def\PY@tc##1{\textcolor[rgb]{0.00,0.50,0.00}{##1}}}
\expandafter\def\csname PY@tok@nc\endcsname{\let\PY@bf=\textbf\def\PY@tc##1{\textcolor[rgb]{0.00,0.00,1.00}{##1}}}
\expandafter\def\csname PY@tok@nd\endcsname{\def\PY@tc##1{\textcolor[rgb]{0.67,0.13,1.00}{##1}}}
\expandafter\def\csname PY@tok@ne\endcsname{\let\PY@bf=\textbf\def\PY@tc##1{\textcolor[rgb]{0.82,0.25,0.23}{##1}}}
\expandafter\def\csname PY@tok@nf\endcsname{\def\PY@tc##1{\textcolor[rgb]{0.00,0.00,1.00}{##1}}}
\expandafter\def\csname PY@tok@si\endcsname{\let\PY@bf=\textbf\def\PY@tc##1{\textcolor[rgb]{0.73,0.40,0.53}{##1}}}
\expandafter\def\csname PY@tok@s2\endcsname{\def\PY@tc##1{\textcolor[rgb]{0.73,0.13,0.13}{##1}}}
\expandafter\def\csname PY@tok@nt\endcsname{\let\PY@bf=\textbf\def\PY@tc##1{\textcolor[rgb]{0.00,0.50,0.00}{##1}}}
\expandafter\def\csname PY@tok@nv\endcsname{\def\PY@tc##1{\textcolor[rgb]{0.10,0.09,0.49}{##1}}}
\expandafter\def\csname PY@tok@s1\endcsname{\def\PY@tc##1{\textcolor[rgb]{0.73,0.13,0.13}{##1}}}
\expandafter\def\csname PY@tok@dl\endcsname{\def\PY@tc##1{\textcolor[rgb]{0.73,0.13,0.13}{##1}}}
\expandafter\def\csname PY@tok@ch\endcsname{\let\PY@it=\textit\def\PY@tc##1{\textcolor[rgb]{0.25,0.50,0.50}{##1}}}
\expandafter\def\csname PY@tok@m\endcsname{\def\PY@tc##1{\textcolor[rgb]{0.40,0.40,0.40}{##1}}}
\expandafter\def\csname PY@tok@gp\endcsname{\let\PY@bf=\textbf\def\PY@tc##1{\textcolor[rgb]{0.00,0.00,0.50}{##1}}}
\expandafter\def\csname PY@tok@sh\endcsname{\def\PY@tc##1{\textcolor[rgb]{0.73,0.13,0.13}{##1}}}
\expandafter\def\csname PY@tok@ow\endcsname{\let\PY@bf=\textbf\def\PY@tc##1{\textcolor[rgb]{0.67,0.13,1.00}{##1}}}
\expandafter\def\csname PY@tok@sx\endcsname{\def\PY@tc##1{\textcolor[rgb]{0.00,0.50,0.00}{##1}}}
\expandafter\def\csname PY@tok@bp\endcsname{\def\PY@tc##1{\textcolor[rgb]{0.00,0.50,0.00}{##1}}}
\expandafter\def\csname PY@tok@c1\endcsname{\let\PY@it=\textit\def\PY@tc##1{\textcolor[rgb]{0.25,0.50,0.50}{##1}}}
\expandafter\def\csname PY@tok@fm\endcsname{\def\PY@tc##1{\textcolor[rgb]{0.00,0.00,1.00}{##1}}}
\expandafter\def\csname PY@tok@o\endcsname{\def\PY@tc##1{\textcolor[rgb]{0.40,0.40,0.40}{##1}}}
\expandafter\def\csname PY@tok@kc\endcsname{\let\PY@bf=\textbf\def\PY@tc##1{\textcolor[rgb]{0.00,0.50,0.00}{##1}}}
\expandafter\def\csname PY@tok@c\endcsname{\let\PY@it=\textit\def\PY@tc##1{\textcolor[rgb]{0.25,0.50,0.50}{##1}}}
\expandafter\def\csname PY@tok@mf\endcsname{\def\PY@tc##1{\textcolor[rgb]{0.40,0.40,0.40}{##1}}}
\expandafter\def\csname PY@tok@err\endcsname{\def\PY@bc##1{\setlength{\fboxsep}{0pt}\fcolorbox[rgb]{1.00,0.00,0.00}{1,1,1}{\strut ##1}}}
\expandafter\def\csname PY@tok@mb\endcsname{\def\PY@tc##1{\textcolor[rgb]{0.40,0.40,0.40}{##1}}}
\expandafter\def\csname PY@tok@ss\endcsname{\def\PY@tc##1{\textcolor[rgb]{0.10,0.09,0.49}{##1}}}
\expandafter\def\csname PY@tok@sr\endcsname{\def\PY@tc##1{\textcolor[rgb]{0.73,0.40,0.53}{##1}}}
\expandafter\def\csname PY@tok@mo\endcsname{\def\PY@tc##1{\textcolor[rgb]{0.40,0.40,0.40}{##1}}}
\expandafter\def\csname PY@tok@kd\endcsname{\let\PY@bf=\textbf\def\PY@tc##1{\textcolor[rgb]{0.00,0.50,0.00}{##1}}}
\expandafter\def\csname PY@tok@mi\endcsname{\def\PY@tc##1{\textcolor[rgb]{0.40,0.40,0.40}{##1}}}
\expandafter\def\csname PY@tok@kn\endcsname{\let\PY@bf=\textbf\def\PY@tc##1{\textcolor[rgb]{0.00,0.50,0.00}{##1}}}
\expandafter\def\csname PY@tok@cpf\endcsname{\let\PY@it=\textit\def\PY@tc##1{\textcolor[rgb]{0.25,0.50,0.50}{##1}}}
\expandafter\def\csname PY@tok@kr\endcsname{\let\PY@bf=\textbf\def\PY@tc##1{\textcolor[rgb]{0.00,0.50,0.00}{##1}}}
\expandafter\def\csname PY@tok@s\endcsname{\def\PY@tc##1{\textcolor[rgb]{0.73,0.13,0.13}{##1}}}
\expandafter\def\csname PY@tok@kp\endcsname{\def\PY@tc##1{\textcolor[rgb]{0.00,0.50,0.00}{##1}}}
\expandafter\def\csname PY@tok@w\endcsname{\def\PY@tc##1{\textcolor[rgb]{0.73,0.73,0.73}{##1}}}
\expandafter\def\csname PY@tok@kt\endcsname{\def\PY@tc##1{\textcolor[rgb]{0.69,0.00,0.25}{##1}}}
\expandafter\def\csname PY@tok@sc\endcsname{\def\PY@tc##1{\textcolor[rgb]{0.73,0.13,0.13}{##1}}}
\expandafter\def\csname PY@tok@sb\endcsname{\def\PY@tc##1{\textcolor[rgb]{0.73,0.13,0.13}{##1}}}
\expandafter\def\csname PY@tok@sa\endcsname{\def\PY@tc##1{\textcolor[rgb]{0.73,0.13,0.13}{##1}}}
\expandafter\def\csname PY@tok@k\endcsname{\let\PY@bf=\textbf\def\PY@tc##1{\textcolor[rgb]{0.00,0.50,0.00}{##1}}}
\expandafter\def\csname PY@tok@se\endcsname{\let\PY@bf=\textbf\def\PY@tc##1{\textcolor[rgb]{0.73,0.40,0.13}{##1}}}
\expandafter\def\csname PY@tok@sd\endcsname{\let\PY@it=\textit\def\PY@tc##1{\textcolor[rgb]{0.73,0.13,0.13}{##1}}}

\def\PYZbs{\char`\\}
\def\PYZus{\char`\_}
\def\PYZob{\char`\{}
\def\PYZcb{\char`\}}
\def\PYZca{\char`\^}
\def\PYZam{\char`\&}
\def\PYZlt{\char`\<}
\def\PYZgt{\char`\>}
\def\PYZsh{\char`\#}
\def\PYZpc{\char`\%}
\def\PYZdl{\char`\$}
\def\PYZhy{\char`\-}
\def\PYZsq{\char`\'}
\def\PYZdq{\char`\"}
\def\PYZti{\char`\~}
% for compatibility with earlier versions
\def\PYZat{@}
\def\PYZlb{[}
\def\PYZrb{]}
\makeatother


    % Exact colors from NB
    \definecolor{incolor}{rgb}{0.0, 0.0, 0.5}
    \definecolor{outcolor}{rgb}{0.545, 0.0, 0.0}



    
    % Prevent overflowing lines due to hard-to-break entities
    \sloppy 
    % Setup hyperref package
    \hypersetup{
      breaklinks=true,  % so long urls are correctly broken across lines
      colorlinks=true,
      urlcolor=urlcolor,
      linkcolor=linkcolor,
      citecolor=citecolor,
      }
    % Slightly bigger margins than the latex defaults
    
    \geometry{verbose,tmargin=1in,bmargin=1in,lmargin=1in,rmargin=1in}
    
    

    \begin{document}
    
    
    \maketitle
    
    

    
    \section{QUESTIONS}\label{questions}

In the TMDb movie data analysis we will investige the following data

\begin{itemize}
\tightlist
\item
  How the popularity of the movie varies across different years. We can
  also get the top 5 year and the lowest 5 year according to the
  popularity of movies.
\item
  How the budget of the movies varies across different years.
\item
  We will perform the hypothesis test where we try to investigate that
  wether the vote average of action movies are more than the other
  movies.
\end{itemize}

    \begin{Verbatim}[commandchars=\\\{\}]
{\color{incolor}In [{\color{incolor}205}]:} \PY{l+s+sd}{\PYZdq{}\PYZdq{}\PYZdq{}}
          \PY{l+s+sd}{ created by deepak }
          \PY{l+s+sd}{ description importing the libraries used for the movie data analysis}
          \PY{l+s+sd}{\PYZdq{}\PYZdq{}\PYZdq{}}
          \PY{k+kn}{import} \PY{n+nn}{numpy} \PY{k+kn}{as} \PY{n+nn}{np}
          \PY{k+kn}{import} \PY{n+nn}{pandas} \PY{k+kn}{as} \PY{n+nn}{pd}
          \PY{k+kn}{import} \PY{n+nn}{matplotlib.pyplot} \PY{k+kn}{as} \PY{n+nn}{plt}
          \PY{k+kn}{import} \PY{n+nn}{seaborn} \PY{k+kn}{as} \PY{n+nn}{sns}
          \PY{o}{\PYZpc{}}\PY{k}{pylab} inline
\end{Verbatim}


    \begin{Verbatim}[commandchars=\\\{\}]
Populating the interactive namespace from numpy and matplotlib

    \end{Verbatim}

    \begin{Verbatim}[commandchars=\\\{\}]
{\color{incolor}In [{\color{incolor}206}]:} \PY{l+s+sd}{\PYZdq{}\PYZdq{}\PYZdq{}}
          \PY{l+s+sd}{created by deepak }
          \PY{l+s+sd}{ description data frame for reading all the movie list created and seeing the top data on the list using the head function}
          \PY{l+s+sd}{\PYZdq{}\PYZdq{}\PYZdq{}}
          \PY{n}{movie\PYZus{}df}\PY{o}{=}\PY{n}{pd}\PY{o}{.}\PY{n}{read\PYZus{}csv}\PY{p}{(}\PY{l+s+s1}{\PYZsq{}}\PY{l+s+s1}{tmdb\PYZhy{}movies.csv}\PY{l+s+s1}{\PYZsq{}}\PY{p}{)}
          \PY{n}{movie\PYZus{}df}\PY{o}{.}\PY{n}{head}\PY{p}{(}\PY{p}{)}
\end{Verbatim}


\begin{Verbatim}[commandchars=\\\{\}]
{\color{outcolor}Out[{\color{outcolor}206}]:}        id    imdb\_id  popularity     budget     revenue  \textbackslash{}
          0  135397  tt0369610   32.985763  150000000  1513528810   
          1   76341  tt1392190   28.419936  150000000   378436354   
          2  262500  tt2908446   13.112507  110000000   295238201   
          3  140607  tt2488496   11.173104  200000000  2068178225   
          4  168259  tt2820852    9.335014  190000000  1506249360   
          
                           original\_title  \textbackslash{}
          0                Jurassic World   
          1            Mad Max: Fury Road   
          2                     Insurgent   
          3  Star Wars: The Force Awakens   
          4                     Furious 7   
          
                                                          cast  \textbackslash{}
          0  Chris Pratt|Bryce Dallas Howard|Irrfan Khan|Vi{\ldots}   
          1  Tom Hardy|Charlize Theron|Hugh Keays-Byrne|Nic{\ldots}   
          2  Shailene Woodley|Theo James|Kate Winslet|Ansel{\ldots}   
          3  Harrison Ford|Mark Hamill|Carrie Fisher|Adam D{\ldots}   
          4  Vin Diesel|Paul Walker|Jason Statham|Michelle {\ldots}   
          
                                                      homepage          director  \textbackslash{}
          0                      http://www.jurassicworld.com/   Colin Trevorrow   
          1                        http://www.madmaxmovie.com/     George Miller   
          2     http://www.thedivergentseries.movie/\#insurgent  Robert Schwentke   
          3  http://www.starwars.com/films/star-wars-episod{\ldots}       J.J. Abrams   
          4                           http://www.furious7.com/         James Wan   
          
                                   tagline      {\ldots}       \textbackslash{}
          0              The park is open.      {\ldots}        
          1             What a Lovely Day.      {\ldots}        
          2     One Choice Can Destroy You      {\ldots}        
          3  Every generation has a story.      {\ldots}        
          4            Vengeance Hits Home      {\ldots}        
          
                                                      overview runtime  \textbackslash{}
          0  Twenty-two years after the events of Jurassic {\ldots}     124   
          1  An apocalyptic story set in the furthest reach{\ldots}     120   
          2  Beatrice Prior must confront her inner demons {\ldots}     119   
          3  Thirty years after defeating the Galactic Empi{\ldots}     136   
          4  Deckard Shaw seeks revenge against Dominic Tor{\ldots}     137   
          
                                                genres  \textbackslash{}
          0  Action|Adventure|Science Fiction|Thriller   
          1  Action|Adventure|Science Fiction|Thriller   
          2         Adventure|Science Fiction|Thriller   
          3   Action|Adventure|Science Fiction|Fantasy   
          4                      Action|Crime|Thriller   
          
                                          production\_companies release\_date vote\_count  \textbackslash{}
          0  Universal Studios|Amblin Entertainment|Legenda{\ldots}   06-09-2015       5562   
          1  Village Roadshow Pictures|Kennedy Miller Produ{\ldots}      5/13/15       6185   
          2  Summit Entertainment|Mandeville Films|Red Wago{\ldots}      3/18/15       2480   
          3          Lucasfilm|Truenorth Productions|Bad Robot     12/15/15       5292   
          4  Universal Pictures|Original Film|Media Rights {\ldots}   04-01-2015       2947   
          
             vote\_average  release\_year   budget\_adj   revenue\_adj  
          0           6.5          2015  137999939.3  1.392446e+09  
          1           7.1          2015  137999939.3  3.481613e+08  
          2           6.3          2015  101199955.5  2.716190e+08  
          3           7.5          2015  183999919.0  1.902723e+09  
          4           7.3          2015  174799923.1  1.385749e+09  
          
          [5 rows x 21 columns]
\end{Verbatim}
            
    \begin{Verbatim}[commandchars=\\\{\}]
{\color{incolor}In [{\color{incolor}207}]:} \PY{l+s+sd}{\PYZdq{}\PYZdq{}\PYZdq{}}
          \PY{l+s+sd}{created by deepak}
          \PY{l+s+sd}{description dataframe to give the mean of movies according to the release year.}
          \PY{l+s+sd}{Here we are trying to group all the movies and group them according to their release year and finally taking the mean of the data.}
          \PY{l+s+sd}{\PYZdq{}\PYZdq{}\PYZdq{}}
          \PY{n}{data\PYZus{}movie}\PY{o}{=}\PY{n}{movie\PYZus{}df}\PY{o}{.}\PY{n}{groupby}\PY{p}{(}\PY{p}{[}\PY{l+s+s1}{\PYZsq{}}\PY{l+s+s1}{release\PYZus{}year}\PY{l+s+s1}{\PYZsq{}}\PY{p}{]}\PY{p}{,}\PY{n}{as\PYZus{}index}\PY{o}{=}\PY{n+nb+bp}{False}\PY{p}{)}\PY{o}{.}\PY{n}{mean}\PY{p}{(}\PY{p}{)}
          \PY{n}{data\PYZus{}movie}\PY{o}{.}\PY{n}{head}\PY{p}{(}\PY{p}{)}
\end{Verbatim}


\begin{Verbatim}[commandchars=\\\{\}]
{\color{outcolor}Out[{\color{outcolor}207}]:}    release\_year            id  popularity        budget       revenue  \textbackslash{}
          0          1960  15715.281250    0.458932  6.892796e+05  4.531406e+06   
          1          1961  18657.000000    0.422827  1.488290e+06  1.089420e+07   
          2          1962  17001.062500    0.454783  1.710066e+06  6.736870e+06   
          3          1963  16556.000000    0.502706  2.156809e+06  5.511911e+06   
          4          1964  17379.571429    0.412428  9.400753e+05  8.118614e+06   
          
                runtime  vote\_count  vote\_average    budget\_adj   revenue\_adj  
          0  110.656250   77.531250      6.325000  5.082036e+06  3.340991e+07  
          1  119.419355   77.580645      6.374194  1.085687e+07  7.947167e+07  
          2  124.343750   74.750000      6.343750  1.232693e+07  4.856238e+07  
          3  111.323529   82.823529      6.329412  1.535687e+07  3.924580e+07  
          4  109.214286   74.690476      6.211905  6.608980e+06  5.707603e+07  
\end{Verbatim}
            
    \section{Top 5 popular year}\label{top-5-popular-year}

Here we are creating the dataframe from the data\_movie dataframe and
sorting the years according to the popularity in descending order. Then
we are displalying the top five years where the mean popularity was
higher than the other years.

    \begin{Verbatim}[commandchars=\\\{\}]
{\color{incolor}In [{\color{incolor}208}]:} \PY{n}{data\PYZus{}movie\PYZus{}sorted}\PY{o}{=}\PY{n}{data\PYZus{}movie}\PY{o}{.}\PY{n}{sort\PYZus{}values}\PY{p}{(}\PY{l+s+s1}{\PYZsq{}}\PY{l+s+s1}{popularity}\PY{l+s+s1}{\PYZsq{}}\PY{p}{,}\PY{n}{ascending}\PY{o}{=}\PY{n+nb+bp}{False}\PY{p}{)}
          \PY{n}{data\PYZus{}movie\PYZus{}sorted}\PY{o}{.}\PY{n}{head}\PY{p}{(}\PY{l+m+mi}{5}\PY{p}{)}
\end{Verbatim}


\begin{Verbatim}[commandchars=\\\{\}]
{\color{outcolor}Out[{\color{outcolor}208}]:}     release\_year             id  popularity        budget       revenue  \textbackslash{}
          55          2015  296762.012719    1.030657  1.207718e+07  4.254762e+07   
          54          2014  243004.421429    0.887268  1.131999e+07  3.475879e+07   
          44          2004   14554.231270    0.722438  2.335616e+07  5.470301e+07   
          43          2003   19671.323843    0.719083  2.220590e+07  5.387275e+07   
          37          1997   15234.760417    0.712003  2.474524e+07  5.549569e+07   
          
                 runtime  vote\_count  vote\_average    budget\_adj   revenue\_adj  
          55   96.375199  290.019078      5.885692  1.111100e+07  3.914379e+07  
          54   98.331429  294.660000      5.920714  1.042674e+07  3.201601e+07  
          44  105.364821  257.980456      5.988599  2.696341e+07  6.315163e+07  
          43  100.679715  243.505338      5.930961  2.632182e+07  6.385821e+07  
          37  106.505208  213.463542      5.988542  3.361543e+07  7.538870e+07  
\end{Verbatim}
            
    \section{Lowest 5 popular year}\label{lowest-5-popular-year}

Here we are displalying the lowest five years where the mean popularity
was higher than the other years.

    \begin{Verbatim}[commandchars=\\\{\}]
{\color{incolor}In [{\color{incolor}209}]:} \PY{n}{data\PYZus{}movie\PYZus{}sorted}\PY{o}{.}\PY{n}{tail}\PY{p}{(}\PY{l+m+mi}{5}\PY{p}{)}
\end{Verbatim}


\begin{Verbatim}[commandchars=\\\{\}]
{\color{outcolor}Out[{\color{outcolor}209}]:}     release\_year            id  popularity        budget       revenue  \textbackslash{}
          18          1978  21710.769231    0.413314  3.215339e+06  2.107353e+07   
          4           1964  17379.571429    0.412428  9.400753e+05  8.118614e+06   
          5           1965  17564.714286    0.342587  2.005860e+06  1.308805e+07   
          10          1970  18339.195122    0.341700  3.096755e+06  1.366395e+07   
          6           1966  16514.000000    0.304112  1.251191e+06  1.842102e+06   
          
                 runtime  vote\_count  vote\_average    budget\_adj   revenue\_adj  
          18  110.076923   75.353846      6.130769  1.074791e+07  7.044251e+07  
          4   109.214286   74.690476      6.211905  6.608980e+06  5.707603e+07  
          5   118.171429   52.000000      6.194286  1.388168e+07  9.057670e+07  
          10  112.048780   49.048780      6.417073  1.739252e+07  7.674178e+07  
          6   106.891304   31.739130      6.128261  8.405522e+06  1.237527e+07  
\end{Verbatim}
            
    \section{Scatter Plot (Release Year,
Popularity)}\label{scatter-plot-release-year-popularity}

The graph plots the average popularity of the movies in different years.
The graph scale as year in x-axis and the popularity on the scale of 0.1
from 0.3 to 1 on the y-axis

    \begin{Verbatim}[commandchars=\\\{\}]
{\color{incolor}In [{\color{incolor}210}]:} \PY{n}{fig} \PY{o}{=} \PY{n}{plt}\PY{o}{.}\PY{n}{figure}\PY{p}{(}\PY{p}{)}
          \PY{n}{fig}\PY{o}{.}\PY{n}{suptitle}\PY{p}{(}\PY{l+s+s1}{\PYZsq{}}\PY{l+s+s1}{Popularity per Year}\PY{l+s+s1}{\PYZsq{}}\PY{p}{,} \PY{n}{fontsize}\PY{o}{=}\PY{l+m+mi}{20}\PY{p}{)}
          
          \PY{n}{plt}\PY{o}{.}\PY{n}{xlabel}\PY{p}{(}\PY{l+s+s1}{\PYZsq{}}\PY{l+s+s1}{year}\PY{l+s+s1}{\PYZsq{}}\PY{p}{,} \PY{n}{fontsize}\PY{o}{=}\PY{l+m+mi}{16}\PY{p}{)}
          \PY{n}{plt}\PY{o}{.}\PY{n}{ylabel}\PY{p}{(}\PY{l+s+s1}{\PYZsq{}}\PY{l+s+s1}{popularity}\PY{l+s+s1}{\PYZsq{}}\PY{p}{,} \PY{n}{fontsize}\PY{o}{=}\PY{l+m+mi}{16}\PY{p}{)}
          \PY{n}{plt}\PY{o}{.}\PY{n}{scatter}\PY{p}{(}\PY{n}{data\PYZus{}movie}\PY{p}{[}\PY{l+s+s1}{\PYZsq{}}\PY{l+s+s1}{release\PYZus{}year}\PY{l+s+s1}{\PYZsq{}}\PY{p}{]}\PY{p}{,}\PY{n}{data\PYZus{}movie}\PY{p}{[}\PY{l+s+s1}{\PYZsq{}}\PY{l+s+s1}{popularity}\PY{l+s+s1}{\PYZsq{}}\PY{p}{]}\PY{p}{)}
\end{Verbatim}


\begin{Verbatim}[commandchars=\\\{\}]
{\color{outcolor}Out[{\color{outcolor}210}]:} <matplotlib.collections.PathCollection at 0x1437e860>
\end{Verbatim}
            
    \begin{center}
    \adjustimage{max size={0.9\linewidth}{0.9\paperheight}}{output_9_1.png}
    \end{center}
    { \hspace*{\fill} \\}
    
    \section{Conclusions}\label{conclusions}

From the above graph we can conclude that the average popularity of the
movies have increased over the time.

Limitations -

The popularity of the movies are not sole dependent on the year as they
can be furthur investigated under the genre of the movie.

The analysis can be furthur made interested if the age of the viewers
can be provided which makes it easier to understand that which age group
was most active and furthur investigation could be done.

    \section{Bar Graph(Release Year,
Budget)}\label{bar-graphrelease-year-budget}

The bar graph shows the average budget of the movies across different
year. The graph shows the year on the x-axis and the budget of the
movies on the y-axis.

    \begin{Verbatim}[commandchars=\\\{\}]
{\color{incolor}In [{\color{incolor}211}]:} \PY{n}{fig} \PY{o}{=} \PY{n}{plt}\PY{o}{.}\PY{n}{figure}\PY{p}{(}\PY{p}{)}
          \PY{n}{fig}\PY{o}{.}\PY{n}{suptitle}\PY{p}{(}\PY{l+s+s1}{\PYZsq{}}\PY{l+s+s1}{Budget}\PY{l+s+s1}{\PYZsq{}}\PY{p}{,} \PY{n}{fontsize}\PY{o}{=}\PY{l+m+mi}{20}\PY{p}{)}
          
          \PY{n}{plt}\PY{o}{.}\PY{n}{xlabel}\PY{p}{(}\PY{l+s+s1}{\PYZsq{}}\PY{l+s+s1}{year}\PY{l+s+s1}{\PYZsq{}}\PY{p}{,} \PY{n}{fontsize}\PY{o}{=}\PY{l+m+mi}{16}\PY{p}{)}
          \PY{n}{plt}\PY{o}{.}\PY{n}{ylabel}\PY{p}{(}\PY{l+s+s1}{\PYZsq{}}\PY{l+s+s1}{dollars}\PY{l+s+s1}{\PYZsq{}}\PY{p}{,} \PY{n}{fontsize}\PY{o}{=}\PY{l+m+mi}{16}\PY{p}{)}
          \PY{n}{plt}\PY{o}{.}\PY{n}{bar}\PY{p}{(}\PY{n}{data\PYZus{}movie}\PY{p}{[}\PY{l+s+s1}{\PYZsq{}}\PY{l+s+s1}{release\PYZus{}year}\PY{l+s+s1}{\PYZsq{}}\PY{p}{]}\PY{p}{,}\PY{n}{data\PYZus{}movie}\PY{p}{[}\PY{l+s+s1}{\PYZsq{}}\PY{l+s+s1}{budget}\PY{l+s+s1}{\PYZsq{}}\PY{p}{]}\PY{p}{)}
\end{Verbatim}


\begin{Verbatim}[commandchars=\\\{\}]
{\color{outcolor}Out[{\color{outcolor}211}]:} <Container object of 56 artists>
\end{Verbatim}
            
    \begin{center}
    \adjustimage{max size={0.9\linewidth}{0.9\paperheight}}{output_12_1.png}
    \end{center}
    { \hspace*{\fill} \\}
    
    \section{Conclusion}\label{conclusion}

The graph shows that the average budget of the movies have increased
from mid 90's in the preceedings year.

Limitaions : There could be limitaions as over the time the production
time must have incresed and is dependent on various variables over time.
Good analysis can have the data of some of the factors and the analysis
can be more accurate to investigate the reasons for the increase in
budget. Eg- salary of actors, set cost.

The action movies can be more costly as compared to a romantic or comedy
movie and the technology have also been evolved as compared to early
90's. So in this analysis these consideration are neglected and the
analysis are made on sole basis of the the average budget of the movies
in each year.

    \# Data Wrangling And Cleaning The data contains the fields where the
budget,revenue are not provided, so we are removing the lower bound
outlieres from our data frame. We are taking the movie\_budget\_df
datdaframe and only taking out the movie with budget, revenue and the
vote count more than 100. The reason for this is that it will eliminate
the values where the value is not provided or the value is given zero.
Also it helps to eliminate the outliers from our dataset. Finally we are
using the head function on the dataframe to see the top five rows of our
dataframe.

    \begin{Verbatim}[commandchars=\\\{\}]
{\color{incolor}In [{\color{incolor}212}]:} \PY{n}{movie\PYZus{}budget\PYZus{}df}\PY{o}{=}\PY{n}{movie\PYZus{}df}\PY{p}{[}\PY{n}{movie\PYZus{}df}\PY{o}{.}\PY{n}{budget}\PY{o}{\PYZgt{}}\PY{l+m+mi}{100}\PY{p}{]}
          \PY{n}{movie\PYZus{}budget\PYZus{}df}\PY{o}{=}\PY{n}{movie\PYZus{}budget\PYZus{}df}\PY{p}{[}\PY{n}{movie\PYZus{}budget\PYZus{}df}\PY{o}{.}\PY{n}{revenue}\PY{o}{\PYZgt{}}\PY{l+m+mi}{100}\PY{p}{]}
          \PY{n}{movie\PYZus{}budget\PYZus{}df}\PY{o}{=}\PY{n}{movie\PYZus{}budget\PYZus{}df}\PY{p}{[}\PY{n}{movie\PYZus{}budget\PYZus{}df}\PY{o}{.}\PY{n}{vote\PYZus{}count}\PY{o}{\PYZgt{}}\PY{l+m+mi}{100}\PY{p}{]}
          
          \PY{n}{movie\PYZus{}budget\PYZus{}df}\PY{o}{.}\PY{n}{head}\PY{p}{(}\PY{p}{)}
\end{Verbatim}


\begin{Verbatim}[commandchars=\\\{\}]
{\color{outcolor}Out[{\color{outcolor}212}]:}        id    imdb\_id  popularity     budget     revenue  \textbackslash{}
          0  135397  tt0369610   32.985763  150000000  1513528810   
          1   76341  tt1392190   28.419936  150000000   378436354   
          2  262500  tt2908446   13.112507  110000000   295238201   
          3  140607  tt2488496   11.173104  200000000  2068178225   
          4  168259  tt2820852    9.335014  190000000  1506249360   
          
                           original\_title  \textbackslash{}
          0                Jurassic World   
          1            Mad Max: Fury Road   
          2                     Insurgent   
          3  Star Wars: The Force Awakens   
          4                     Furious 7   
          
                                                          cast  \textbackslash{}
          0  Chris Pratt|Bryce Dallas Howard|Irrfan Khan|Vi{\ldots}   
          1  Tom Hardy|Charlize Theron|Hugh Keays-Byrne|Nic{\ldots}   
          2  Shailene Woodley|Theo James|Kate Winslet|Ansel{\ldots}   
          3  Harrison Ford|Mark Hamill|Carrie Fisher|Adam D{\ldots}   
          4  Vin Diesel|Paul Walker|Jason Statham|Michelle {\ldots}   
          
                                                      homepage          director  \textbackslash{}
          0                      http://www.jurassicworld.com/   Colin Trevorrow   
          1                        http://www.madmaxmovie.com/     George Miller   
          2     http://www.thedivergentseries.movie/\#insurgent  Robert Schwentke   
          3  http://www.starwars.com/films/star-wars-episod{\ldots}       J.J. Abrams   
          4                           http://www.furious7.com/         James Wan   
          
                                   tagline      {\ldots}       \textbackslash{}
          0              The park is open.      {\ldots}        
          1             What a Lovely Day.      {\ldots}        
          2     One Choice Can Destroy You      {\ldots}        
          3  Every generation has a story.      {\ldots}        
          4            Vengeance Hits Home      {\ldots}        
          
                                                      overview runtime  \textbackslash{}
          0  Twenty-two years after the events of Jurassic {\ldots}     124   
          1  An apocalyptic story set in the furthest reach{\ldots}     120   
          2  Beatrice Prior must confront her inner demons {\ldots}     119   
          3  Thirty years after defeating the Galactic Empi{\ldots}     136   
          4  Deckard Shaw seeks revenge against Dominic Tor{\ldots}     137   
          
                                                genres  \textbackslash{}
          0  Action|Adventure|Science Fiction|Thriller   
          1  Action|Adventure|Science Fiction|Thriller   
          2         Adventure|Science Fiction|Thriller   
          3   Action|Adventure|Science Fiction|Fantasy   
          4                      Action|Crime|Thriller   
          
                                          production\_companies release\_date vote\_count  \textbackslash{}
          0  Universal Studios|Amblin Entertainment|Legenda{\ldots}   06-09-2015       5562   
          1  Village Roadshow Pictures|Kennedy Miller Produ{\ldots}      5/13/15       6185   
          2  Summit Entertainment|Mandeville Films|Red Wago{\ldots}      3/18/15       2480   
          3          Lucasfilm|Truenorth Productions|Bad Robot     12/15/15       5292   
          4  Universal Pictures|Original Film|Media Rights {\ldots}   04-01-2015       2947   
          
             vote\_average  release\_year   budget\_adj   revenue\_adj  
          0           6.5          2015  137999939.3  1.392446e+09  
          1           7.1          2015  137999939.3  3.481613e+08  
          2           6.3          2015  101199955.5  2.716190e+08  
          3           7.5          2015  183999919.0  1.902723e+09  
          4           7.3          2015  174799923.1  1.385749e+09  
          
          [5 rows x 21 columns]
\end{Verbatim}
            
    \section{Standarized Function}\label{standarized-function}

This method takes the list and calculates its mean, standard deviation
and returns the list with the standarized values in it

    \begin{Verbatim}[commandchars=\\\{\}]
{\color{incolor}In [{\color{incolor}213}]:} \PY{k}{def} \PY{n+nf}{standard\PYZus{}list}\PY{p}{(}\PY{n}{list\PYZus{}to\PYZus{}standardize}\PY{p}{)}\PY{p}{:}
              \PY{l+s+sd}{\PYZdq{}\PYZdq{}\PYZdq{}}
          \PY{l+s+sd}{    created by deepak}
          \PY{l+s+sd}{    description the method taked the list and send the list after standardizing it by performing the vectorized operations to it.}
          \PY{l+s+sd}{    \PYZdq{}\PYZdq{}\PYZdq{}}
              \PY{n}{mean}\PY{o}{=}\PY{n}{list\PYZus{}to\PYZus{}standardize}\PY{o}{.}\PY{n}{mean}\PY{p}{(}\PY{p}{)}
              \PY{n}{sd}\PY{o}{=}\PY{n}{list\PYZus{}to\PYZus{}standardize}\PY{o}{.}\PY{n}{std}\PY{p}{(}\PY{p}{)}
              \PY{k}{return} \PY{p}{(}\PY{n}{list\PYZus{}to\PYZus{}standardize}\PY{o}{\PYZhy{}}\PY{n}{mean}\PY{p}{)}\PY{o}{/}\PY{n}{sd}
\end{Verbatim}


    Here we are creating a one dimension list named as gain list where we
are storing the profit made by each movie by taking out the difference
between the revenue and the budget of the movie we have already omitted
the outliers from the list.

    \begin{Verbatim}[commandchars=\\\{\}]
{\color{incolor}In [{\color{incolor}214}]:} \PY{l+s+sd}{\PYZdq{}\PYZdq{}\PYZdq{}}
          \PY{l+s+sd}{created by deepak}
          \PY{l+s+sd}{description The total gain of the movies by taking the difference of revenue and the budget of the movies using the describe method}
          \PY{l+s+sd}{to decribe the list data.}
          \PY{l+s+sd}{\PYZdq{}\PYZdq{}\PYZdq{}}
          \PY{n}{gain\PYZus{}list}\PY{o}{=}\PY{n}{movie\PYZus{}budget\PYZus{}df}\PY{p}{[}\PY{l+s+s1}{\PYZsq{}}\PY{l+s+s1}{revenue}\PY{l+s+s1}{\PYZsq{}}\PY{p}{]}\PY{o}{\PYZhy{}}\PY{n}{movie\PYZus{}budget\PYZus{}df}\PY{p}{[}\PY{l+s+s1}{\PYZsq{}}\PY{l+s+s1}{budget}\PY{l+s+s1}{\PYZsq{}}\PY{p}{]}
          \PY{n}{gain\PYZus{}list}\PY{o}{.}\PY{n}{describe}\PY{p}{(}\PY{p}{)}
\end{Verbatim}


\begin{Verbatim}[commandchars=\\\{\}]
{\color{outcolor}Out[{\color{outcolor}214}]:} count    2.598000e+03
          mean     1.034365e+08
          std      1.730138e+08
          min     -1.657101e+08
          25\%      9.930207e+06
          50\%      4.994628e+07
          75\%      1.274610e+08
          max      2.544506e+09
          dtype: float64
\end{Verbatim}
            
    Here we are calling the standarized function created above and passing
the gain list to know how much standard deviation away the revenue of
each movie is from the list the list shows the profit in the standard
deviation from the mean profit made by all movies. This list can be used
in various analysis. As of now the it is not used in the analysis.

    \begin{Verbatim}[commandchars=\\\{\}]
{\color{incolor}In [{\color{incolor}215}]:} \PY{n}{standard\PYZus{}list}\PY{p}{(}\PY{n}{gain\PYZus{}list}\PY{p}{)}
\end{Verbatim}


\begin{Verbatim}[commandchars=\\\{\}]
{\color{outcolor}Out[{\color{outcolor}215}]:} 0         7.283190
          1         0.722485
          2         0.472804
          3        10.200006
          4         7.009920
          5         1.702257
          6         1.052905
          7         2.219152
          8         5.660210
          9         3.325007
          10        3.076276
          11       -0.551683
          12       -0.471449
          13        0.301713
          14        5.904727
          15        0.048110
          16        1.007636
          17        1.648225
          18        1.987789
          19        2.237318
          20       -0.487827
          21       -0.241175
          22        1.485745
          23        2.463474
          24        0.011040
          25        2.478956
          26        0.256784
          27        1.267908
          28       -0.202816
          29        0.848605
                     {\ldots}    
          10519    -0.403161
          10589    -0.634722
          10593    -0.499267
          10594     1.906609
          10595     0.296747
          10596    -0.583402
          10597    -0.082574
          10601     0.292924
          10606     0.062345
          10608    -0.359333
          10611    -0.442950
          10648    -0.299174
          10649    -0.505375
          10650    -0.146442
          10653    -0.148176
          10689     0.154665
          10690     0.298114
          10691    -0.033030
          10724    -0.164507
          10725    -0.041197
          10727    -0.359806
          10728    -0.630250
          10755     0.418332
          10756     0.372499
          10757    -0.283714
          10758     0.819481
          10759    -0.194993
          10760     0.201507
          10762    -0.395555
          10770    -0.408849
          Length: 2598, dtype: float64
\end{Verbatim}
            
    \section{Hypothesis Testing of popularity of the action
movie}\label{hypothesis-testing-of-popularity-of-the-action-movie}

Here in this hypothesis test we will try to prove the null hypothesis
wrong that the vote average of action movies is different from the
normal movies by performing the two tail z-test on our data set. The
sample size of our data set is more than 30 and we will assume that we
are considering the total population and not the sample data of the
movies.

Null Hypothesis (H0): The vote average of the action movies is same of
that of the all movies. Alternate Hypothesis(HA): The vote average of
the action movies and other movies are not same.

Test type - A two tail z-test is performed for the analysis for 95\%
confidence inteval.

    \section{Action movie data function}\label{action-movie-data-function}

The function takes the dataframe as input and returns the dataframe
where the genre of the movie contains action on it. \#Data wrangling and
cleaning process - Here we are creating a new empty dataframe as
action\_df which will have the vote\_average in it. - We have uses a for
loop to iterate to get the genres string. - We are checking if the
values of the genre string is not NaN with the if condition. - The genre
string contains the movie genre separated by \textbar{}. - We have
separated the multiple genres with the split method which returns the
array of genres. - Then we are iterating over the one dimension array
using the list to search for the action word. - If the array contains
the action we are adding the average vote to the action\_df dataframe -
In the end we are returning the action\_df dataframe.

    \begin{Verbatim}[commandchars=\\\{\}]
{\color{incolor}In [{\color{incolor}216}]:} \PY{k}{def} \PY{n+nf}{action\PYZus{}movie\PYZus{}list}\PY{p}{(}\PY{n}{genre\PYZus{}df}\PY{p}{)}\PY{p}{:} 
              \PY{l+s+sd}{\PYZdq{}\PYZdq{}\PYZdq{}}
          \PY{l+s+sd}{    created by deepak}
          \PY{l+s+sd}{    description funtion for returning the vote average dataframe from the given dataframe}
          \PY{l+s+sd}{    limitation this method assumes that the dataframe contains the vote\PYZus{}average column and returns only the action genres vote average.}
          \PY{l+s+sd}{    \PYZdq{}\PYZdq{}\PYZdq{}}
              \PY{n}{action\PYZus{}df} \PY{o}{=} \PY{n}{pd}\PY{o}{.}\PY{n}{DataFrame}\PY{p}{(}\PY{n}{columns}\PY{o}{=}\PY{p}{[}\PY{l+s+s1}{\PYZsq{}}\PY{l+s+s1}{vote\PYZus{}average}\PY{l+s+s1}{\PYZsq{}}\PY{p}{]}\PY{p}{)}
              \PY{k}{for} \PY{n}{index}\PY{p}{,} \PY{n}{row} \PY{o+ow}{in} \PY{n}{genre\PYZus{}df}\PY{o}{.}\PY{n}{iterrows}\PY{p}{(}\PY{p}{)}\PY{p}{:}        
                  \PY{n}{genre\PYZus{}string}\PY{o}{=}\PY{n}{row}\PY{p}{[}\PY{l+s+s1}{\PYZsq{}}\PY{l+s+s1}{genres}\PY{l+s+s1}{\PYZsq{}}\PY{p}{]}               
                  \PY{k}{if} \PY{n}{genre\PYZus{}string}\PY{o}{!=}\PY{n}{NaN}\PY{p}{:}
                      \PY{n}{movie\PYZus{}list}\PY{o}{=}\PY{n}{np}\PY{o}{.}\PY{n}{array}\PY{p}{(}\PY{n}{genre\PYZus{}string}\PY{o}{.}\PY{n}{split}\PY{p}{(}\PY{l+s+s1}{\PYZsq{}}\PY{l+s+s1}{|}\PY{l+s+s1}{\PYZsq{}}\PY{p}{)}\PY{p}{)}
                      \PY{k}{for} \PY{n}{i} \PY{o+ow}{in} \PY{n}{np}\PY{o}{.}\PY{n}{nditer}\PY{p}{(}\PY{n}{movie\PYZus{}list}\PY{p}{)}\PY{p}{:}
                          \PY{k}{if}\PY{p}{(}\PY{n}{i}\PY{o}{==}\PY{l+s+s1}{\PYZsq{}}\PY{l+s+s1}{Action}\PY{l+s+s1}{\PYZsq{}}\PY{p}{)}\PY{p}{:}
                              \PY{n}{action\PYZus{}df}\PY{o}{=}\PY{n}{action\PYZus{}df}\PY{o}{.}\PY{n}{append}\PY{p}{(}\PY{p}{\PYZob{}}\PY{l+s+s1}{\PYZsq{}}\PY{l+s+s1}{vote\PYZus{}average}\PY{l+s+s1}{\PYZsq{}}\PY{p}{:}\PY{n}{row}\PY{p}{[}\PY{l+s+s1}{\PYZsq{}}\PY{l+s+s1}{vote\PYZus{}average}\PY{l+s+s1}{\PYZsq{}}\PY{p}{]}\PY{p}{\PYZcb{}}\PY{p}{,} \PY{n}{ignore\PYZus{}index}\PY{o}{=}\PY{n+nb+bp}{True}\PY{p}{)} 
          
              \PY{k}{return} \PY{n}{action\PYZus{}df}
\end{Verbatim}


    \section{Data Wrangling for hypothesis
test}\label{data-wrangling-for-hypothesis-test}

\begin{itemize}
\tightlist
\item
  The genre\_df dataframe contains the rows only where the genres string
  column is not empty and thus eliminating the empty data.
\item
  The action\_movie\_df contains the new dataframe containing the vote
  average of only the action movies.
\item
  We are calculation the mean,standard deviation of all the movies and
  the mean of the action list by taking the vote\_average column from
  the data frame and converting it to the one dimensional list.
\item
  We then calculated the z-score of the analysis.
\end{itemize}

    \begin{Verbatim}[commandchars=\\\{\}]
{\color{incolor}In [{\color{incolor}217}]:} \PY{n}{genre\PYZus{}df}\PY{o}{=}\PY{n}{movie\PYZus{}df}\PY{p}{[}\PY{n}{movie\PYZus{}df}\PY{o}{.}\PY{n}{genres}\PY{o}{\PYZgt{}}\PY{l+m+mi}{0}\PY{p}{]}
          \PY{n}{action\PYZus{}movie\PYZus{}df}\PY{o}{=}\PY{n}{action\PYZus{}movie\PYZus{}list}\PY{p}{(}\PY{n}{genre\PYZus{}df}\PY{p}{)}
          \PY{n}{population\PYZus{}mean}\PY{o}{=}\PY{n}{genre\PYZus{}df}\PY{p}{[}\PY{l+s+s1}{\PYZsq{}}\PY{l+s+s1}{vote\PYZus{}average}\PY{l+s+s1}{\PYZsq{}}\PY{p}{]}\PY{o}{.}\PY{n}{mean}\PY{p}{(}\PY{p}{)}
          \PY{n}{population\PYZus{}std}\PY{o}{=}\PY{n}{genre\PYZus{}df}\PY{p}{[}\PY{l+s+s1}{\PYZsq{}}\PY{l+s+s1}{vote\PYZus{}average}\PY{l+s+s1}{\PYZsq{}}\PY{p}{]}\PY{o}{.}\PY{n}{std}\PY{p}{(}\PY{p}{)}
          \PY{n}{action\PYZus{}movie\PYZus{}mean}\PY{o}{=}\PY{n}{action\PYZus{}movie\PYZus{}df}\PY{p}{[}\PY{l+s+s1}{\PYZsq{}}\PY{l+s+s1}{vote\PYZus{}average}\PY{l+s+s1}{\PYZsq{}}\PY{p}{]}\PY{o}{.}\PY{n}{mean}\PY{p}{(}\PY{p}{)}
          \PY{k}{print} \PY{l+s+s1}{\PYZsq{}}\PY{l+s+s1}{The mean of the population is \PYZob{}\PYZcb{} , the std of population is \PYZob{}\PYZcb{} and the mean of the action movies are \PYZob{}\PYZcb{}}\PY{l+s+s1}{\PYZsq{}}\PY{o}{.}\PY{n}{format}\PY{p}{(}\PY{n}{population\PYZus{}mean}\PY{p}{,}\PY{n}{population\PYZus{}std}\PY{p}{,}\PY{n}{action\PYZus{}movie\PYZus{}mean}\PY{p}{)}
          \PY{n}{z\PYZus{}score}\PY{o}{=}\PY{p}{(}\PY{n}{action\PYZus{}movie\PYZus{}mean}\PY{o}{\PYZhy{}}\PY{n}{population\PYZus{}mean}\PY{p}{)}\PY{o}{/}\PY{n}{population\PYZus{}std}
          \PY{k}{print} \PY{n}{z\PYZus{}score}
\end{Verbatim}


    \begin{Verbatim}[commandchars=\\\{\}]
The mean of the population is 5.97397399244 , the std of population is 0.934260258321 and the mean of the action movies are 5.78742138365
-0.199679486662

    \end{Verbatim}

    \section{Conclusion}\label{conclusion}

From our analysis by performing the z-test on the action movie data we
got the z-score of approx -0.20 and the z-critical value for 95\% two
tail test is +1.96 and -1.96 and clearly our value lies in between this
so we fail to reject the null hypotheis that the vote average of action
movies is different than that of all movies.


    % Add a bibliography block to the postdoc
    
    
    
    \end{document}
